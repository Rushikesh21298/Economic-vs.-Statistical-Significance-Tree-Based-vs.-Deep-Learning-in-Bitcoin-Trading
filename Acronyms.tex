%####
% File name: acronyms.tex
% purpose: a sample file that is used to define abbreviations, acronyms, glossary.
% Created on 05/08/2015, by Wenjia Wang

%#### Notes:
% In this file, you can define all the abbreviations you wish to use in your text.

% 1. Use command "\newacronym" to define an abbreviation/acronym 
% format: \newacronym{label}{name}{description}
% for example:  
\newacronym{cmp}{CMP}{School of Computing Sciences}
\newacronym{uea}{UEA}{University of East Anglia}
\newacronym{loa}{LOA}{List of Abbreviations}

% 2. use command "\gls{label}" or "\Gls{label}" to cite a defined acronym in your .tex file 
% for example: \gls{UEA}
% when used it in the first time, it will produce: University of East Anglia(UEA)
% when used after the 1st time, it will produce: UEA    

% Some more examples defined here.

\newacronym{btc}{BTC}{Bitcoin}
\newacronym{usd}{USD}{united States Doller}
\newacronym{tscv}{TSCV}{time Series Cross Validation}
\newacronym{rmse}{RMSE}{Root Mean Square Error}
\newacronym{lstm}{LSTM}{Long Short Term Memory}
\newacronym{ai}{AI}{Artificial Intelligence}
\newacronym{ml}{ML}{Machine Learning}
\newacronym{rnn}{RNN}{Recurrent Neural Network}
\newacronym{mse}{MSE}{Mean Squared Error}
\newacronym{auc}{AUC}{Area Under Curve}
\newacronym{roc}{ROC}{receiver operating characteristic}
\newacronym{drl}{DRL}{Deep Reinforcement Learning}
% or use the following command to define a glossary term %"\newglossaryentry{label}{name={<name>}, description={<describing blahh blah}}
% e.g

\newglossaryentry{apple}
{
	name={apple}, 
	description={is a kind of sweet fruit}
}
\newglossaryentry{latex}
{
	name=Latex,
	description={is a mark-up text-editing language specially   
		for writing scientific documents}
}
