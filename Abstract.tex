% ------------------------------------------------------------------------
% -*-TeX-*- -*-Hard-*- Smart Wrapping
% ------------------------------------------------------------------------
% Thesis Abstract --------------------------------------------------------

\prefacesection{Abstract}

In the age of digital finance, Bitcoin has risen as a dominant cryptocurrency, characterized by its potential for high returns and significant volatility. This dynamic landscape necessitates advanced predictive tools for traders and investors. Given this backdrop, this dissertation embarked on a journey to compare the economic and statistical significance of tree-based ensembles, notably Random Forest, with that of deep learning models, focusing on \gls{lstm} networks, in the realm of Bitcoin trading.



\smallskip
The initial exploration sets the stage by delving into the historical nuances of Bitcoin trading, elucidating its market dynamics and behavior. With this foundation, the research pivoted to the application of machine learning in financial computing, highlighting both its promises and challenges. A deeper dive was then taken into tree-based ensemble methods, with particular emphasis on Random Forest's role in financial prediction, compared alongside other ensemble methods.

Employing a rigorous methodological framework, models were trained and validated using Time Series Cross-Validation, juxtaposing various training window sizes to ascertain robustness. Key performance metrics such as accuracy, precision, recall, F1-score, and \gls{rmse} served as the evaluative backbone of this study.

\smallskip
Results underscored the nuanced capabilities of both modeling techniques. While Random Forest demonstrated a consistent ability to capture non-linear patterns, \gls{lstm} showcased its strength in handling sequential data inherent in time series. Economic implications derived from these models' predictions revealed potential profitability, though the extent was influenced by the chosen training window.

In summary, while both tree-based ensembles and deep learning models hold promise in predicting Bitcoin price movements, their effectiveness is multifaceted, depending on model parameters and market conditions. This research not only elevates the discourse in Bitcoin trading strategies but also sets the stage for future explorations in the confluence of machine learning and financial forecasting.


% ----------------------------------------------------------------------
